%% Korjaa vastaamaan korkeakouluasi, jos automaattisesti asetettu nimi on 
%% virheellinen 
%%
%% Change the school field to specify your school if the automatically 
%% set name is wrong
% \university{aalto-yliopisto}
% \school{Sähkötekniikan korkeakoulu}

%% VAIN DI/M.Sc.- JA LISENSIAATINTYÖLLE: valitse laitos, 
%% professuuri ja sen professuurikoodi. 
%%
%\department{Radiotieteen ja -tekniikan laitos}
%\professorship{Piiriteoria}
%%

%% Valitse yksi näistä kolmesta
%%
\univdegree{BSc}
%\univdegree{MSc}
%\univdegree{Lic}

%% Oma nimi
%%
\author{oma nimi}

%% Opinnäytteen otsikko tulee tähän ja uudelleen englannin- tai 
%% ruostinkielisen abstraktin yhteydessä. Älä tavuta otsikkoa ja
%% vältä liian pitkää otsikkotekstiä. Jos latex ryhmittelee otsikon
%% huonosti, voit joutua pakottamaan rivinvaihdon \\ kontrollimerkillä.
%% Muista että otsikkoja ei tavuteta! 
%% Jos otsikossa on ja-sana, se ei jää rivin viimeiseksi sanaksi 
%% vaan aloittaa uuden rivin.
%% 
\thesistitle{otsikko}

\place{Espoo}

%% Kandidaatintyön päivämäärä on sen esityspäivämäärä! 
%% 
%TODO
\date{Work in progress! Compiled: \currenttime \space \today}
%\date{27.11.2016}

%% Aaltologo: syntaksi:
%% \uselogo{aaltoRed|aaltoBlue|aaltoYellow|aaltoGray|aaltoGrayScale}{?|!|''}
%% Logon kieli on sama kuin dokumentin kieli
%%
\uselogo{aaltoRed}{''}

%% Tehdään kansilehti
%%
\makecoverpage{}
