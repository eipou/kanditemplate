\documentclass[english,12pt,a4paper,pdftex,elec,utf8]{aaltothesis}

%% Kirjoita y.o. \documentclass optioiksi
%% korkeakoulusi näistä: arts, biz, chem, elec, eng, sci
%% editorisi käyttämä merkkikoodaustapa: utf8, latin1

%% Käytä näitä, jos kirjoitat englanniksi. Katso englanninokset tiedostosta
%% thesistemplate.tex.
%\documentclass[english,12pt,a4paper,pdftex,elec,utf8]{aaltothesis}
%\documentclass[english,12pt,a4paper,dvips]{aaltothesis}

\usepackage[english,finnish]{babel}
\usepackage[fixlanguage]{babelbib}
\usepackage{cite}
\usepackage[section]{placeins} %prevent figures from floating allover
\usepackage{cleveref} % for cref
\usepackage{booktabs} % for midrule etc.
\usepackage{multirow} % for multirow table
\usepackage{siunitx} % for \ang
\usepackage{float}
\usepackage{subfig} % for subfigures
\usepackage[toc,page]{appendix}
\usepackage[a-1b]{pdfx}
\usepackage[yyyymmdd,hhmmss]{datetime}

\usepackage{acro} % for acronyms
\acsetup{
  only-used   = true   
}

\DeclareAcronym{HHI}{
  short = HHI ,
  long  = Human-Human Interaction ,
  class = acronym
}

\DeclareAcronym{HRI}{
  short = HRI ,
  long  = Human-Robot Interaction ,
  class = acronym
}


%% Aloita uudet kappaleet uudelta sivulta
\usepackage{titlesec}
\newcommand{\sectionbreak}{\clearpage}
\usepackage{graphicx}

\usepackage[numbers]{natbib}

\selectbiblanguage{english}
\bibliographystyle{babunsrt}

%% Matematiikan fontteja, symboleja ja muotoiluja lisää, näitä tarvitaan usein 
\usepackage{amsfonts,amssymb,amsbsy,amsmath}

%% Saat pdf-tiedoston viittaukset ja linkit kuntoon seuraavalla paketilla.
\usepackage{hyperref}
\hypersetup{pdfpagemode=UseNone, pdfstartview=FitH,
  colorlinks=true,urlcolor=red,linkcolor=blue,citecolor=black,
  pdftitle={Default Title, Modify},pdfauthor={Your Name},
  pdfkeywords={Modify keywords}}

%% Shows comments in boxes
%\newcommand{\authorcomment}[1]{\fbox{\begin{minipage}{\textwidth}Comment by author:\\#1\end{minipage}}}
%\newcommand{\advisorcomment}[1]{\fbox{\begin{minipage}{\textwidth}Comment by advisor:\\#1\end{minipage}}}

%% Hides comments
\newcommand{\authorcomment}[1]{}
\newcommand{\advisorcomment}[1]{}

%% Kaikki mikä paperille tulostuu, on tämän jälkeen
\begin{document}

%% Kansilehti
%% Korjaa vastaamaan korkeakouluasi, jos automaattisesti asetettu nimi on 
%% virheellinen 
%%
%% Change the school field to specify your school if the automatically 
%% set name is wrong
% \university{aalto-yliopisto}
% \school{Sähkötekniikan korkeakoulu}

%% VAIN DI/M.Sc.- JA LISENSIAATINTYÖLLE: valitse laitos, 
%% professuuri ja sen professuurikoodi. 
%%
%\department{Radiotieteen ja -tekniikan laitos}
%\professorship{Piiriteoria}
%%

%% Valitse yksi näistä kolmesta
%%
\univdegree{BSc}
%\univdegree{MSc}
%\univdegree{Lic}

%% Oma nimi
%%
\author{oma nimi}

%% Opinnäytteen otsikko tulee tähän ja uudelleen englannin- tai 
%% ruostinkielisen abstraktin yhteydessä. Älä tavuta otsikkoa ja
%% vältä liian pitkää otsikkotekstiä. Jos latex ryhmittelee otsikon
%% huonosti, voit joutua pakottamaan rivinvaihdon \\ kontrollimerkillä.
%% Muista että otsikkoja ei tavuteta! 
%% Jos otsikossa on ja-sana, se ei jää rivin viimeiseksi sanaksi 
%% vaan aloittaa uuden rivin.
%% 
\thesistitle{otsikko}

\place{Espoo}

%% Kandidaatintyön päivämäärä on sen esityspäivämäärä! 
%% 
%TODO
\date{Work in progress! Compiled: \currenttime \space \today}
%\date{27.11.2016}

%% Aaltologo: syntaksi:
%% \uselogo{aaltoRed|aaltoBlue|aaltoYellow|aaltoGray|aaltoGrayScale}{?|!|''}
%% Logon kieli on sama kuin dokumentin kieli
%%
\uselogo{aaltoRed}{''}

%% Tehdään kansilehti
%%
\makecoverpage{}

 
%% Tiivistelmä
\keywords{keywords in english}
\degreeprogram{program}
\supervisor{supervisor}
\advisor{advisor}

\begin{abstractpage}[english]
abstract in english

\end{abstractpage}

\newpage
\keywords{avainsanat suomeksi}
\degreeprogram{koulutusohjelma}
\supervisor{vastuuopettaja}
\advisor{ohjaaja}

\begin{abstractpage}[finnish]
lyhyt tiivistelmä suomeksi

\end{abstractpage}




%% Sisällysluettelo
\thesistableofcontents{}

%% Symbolit ja lyhenteet
\mysection{Abbreviations}
\printacronyms[include-classes=acronym,name=Acronyms]


%% Sivulaskurin viilausta opinnäytteen vaatimusten mukaan:
%% Aloitetaan sivunumerointi arabialaisilla numeroilla (ja jätetään
%% leipätekstin ensimmäinen sivu tyhjäksi, 
\cleardoublepage{}
\storeinipagenumber{}
\pagenumbering{arabic}
\setcounter{page}{1}

%% Ensimmäinen sivu tyhjäksi
\thispagestyle{empty}

%% Content

%% Kappaleet tähän
\section{Introduction}

introductiokappale \cite{SGP} (ei käänny jos ei cite) \ac{HRI} duunaa lyhenteet oikein \ac{HRI}.

\section{Conclusions}

conkluusiokappale


\clearpage
%% Lähdeluettelo

\thesisbibliography{}
\bibliography{kandi}

\begin{appendices}

\begin{otherlanguage}{finnish}
	\section{Finnish summary - Suomenkielinen tiivistelmä}

\clearpage

\setcounter{subsection}{-1}
\let\oldsubsection=\thesubsection
\renewcommand{\thesubsection}{\thesection}

\subsection{Otsikko}

Pitkä tiivistelmä suomeksi (3 sivua)

\renewcommand{\thesubsection}{\oldsubsection}

\end{otherlanguage}

\end{appendices}

\end{document}
